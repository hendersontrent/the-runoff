\documentclass{article}
\usepackage{hyperref}

\begin{document}

\title{How similar are the AFL's Top 100 players as measured by average game output?}
\author{Trent Henderson}

\maketitle

\begin{abstract}
T
\end{abstract}

\section{Introduction}
H

\subsection{Data sources}
This analysis uses game data from \href{https://afltables.com/afl/afl_index.html}{AFL Tables} and \href{https://www.footywire.com/}{FootyWire}, pulled using the R package \href{https://github.com/jimmyday12/fitzRoy}{fitzRoy}.

\section{Methodology and Results}
Our methodology involved first aggregating the data to player-level averages for the entire season. We filtered out all games that were finals - as other \href{https://www.orbisantanalytics.com/sport-esports}{research} has highlighted the inherent differences in finals footy compared to regular season games. Averages were then calculated for 5 metrics that would likely apply to players of most positions:

\begin{itemize}
  \item Kicks
  \item Handballs
  \item Marks
  \item Contested marks
  \item Contested possessions
\end{itemize}

These averages were then summed and the top 100 players based on this summation were retained. This was done for two reasons:

\begin{itemize}
  \item To limit the number of nodes and edges to render in the below diagram as they increase exponentially with each player
  \item The top 100 players likely contribute the most to team success and so would be of most immediate importance to understand
\end{itemize}

\subsection{Clustering of players}
The metrics of the top 100 players were entered into a statistical probabilistic clustering model called \href{https://www.sciencedirect.com/science/article/pii/S0001879120300701}{Latent Profile Analysis (LPA)}. LPA was chosen over alternatives such as k-means cluster analysis due to a lack of an formalised view on a potential number of clusters. LPA allows for testing of multiple competing models and assigns each observation a probability of being in each of the computed clusters. The cluster with the highest probability is the most likely category of membership for a given player. This helps us provide some structure to data with multiple variables. 

\subsection{Visualising relationships between players}
T

\textbf{\textit{Interactive network diagram here}}

\end{document}